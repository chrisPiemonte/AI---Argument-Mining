% !TEX encoding = UTF-8
% !TEX TS-program = pdflatex
% !TEX root = ../Tesi.tex
% !TEX spellcheck = it-IT

%************************************************

%************************************************

% sviluppi futuri
L'implementazione del mining degli argumentation framework e del sistema prolog per il ranking delle estensioni presentano molti margini di miglioramento, fra i quali possiamo distinguere in 


\section{Mining}
\label{section:ranking}
Estrarre l'albero della conversazione o il grafo degli utenti può essere ampliato alla analisi di altri social media od addirittura all'estrazione di argomenti da altre fonti web o no. Un'idea potrebbe essere trovare un fonte dati con delle caratteristiche per le quali la creazione del grafo sia implicita nella struttura degli argomenti. I social media presentano una grande opportunità però per motivi di visualizzazione spesso impongono dei vincolo nella struttura dei commenti. 

\subsection {Migliorare generazioni pesi delle relazioni} 
I pesi delle relazioni sono in funzione del sentiment e della similarità dei testi dei due commenti che uniscono. In questo caso è stato utilizzato vader \cite{hutto2014vader} come sentiment analyzer perchè trainato sui social network, ma questo potrebbe cambiare in base alla fonte dati.

La similarità invece è calcolata tramite la creazione di word embedding delle parole attraverso l'algoritmo word2vec \cite{} ed in seguito attraverso la creazione di embedding per l'intero testo attraverso la medie degli embedding delle parole. Questo porta a considerare simili testi che contengono parole simili attraverso la distributional semantic ovvero utilizzate in contesti simili.

L'introduzi




%%%%%%%%%%%%%%%%%%%%%%%%%%%%%%%%%%%%%%%%%%%%%%%%%%%%%%%%%%%%%%%
%%%%%%%%%%%%%%%%%%%%%%%%%%%%%%%%%%%%%%%%%%%%%%%%%%%%%%%%%%%%%%%
%%%%%%%%%%%%%%%%%%%%%%%%%%%%%%%%%%%%%%%%%%%%%%%%%%%%%%%%%%%%%%%
%%%%%%%%%%%%%%%%%%%%%%%%%%%%%%%%%%%%%%%%%%%%%%%%%%%%%%%%%%%%%%%
%%%%%%%%%%%%%%%%%%%%%%%%%%%%%%%%%%%%%%%%%%%%%%%%%%%%%%%%%%%%%%%

\section {Ranking}

\subsection {Nuove funzioni di ranking}

\subsubsection {Node related}
\subsubsection {Graph related}
\subsubsection {Testo dei nodi related}





