% !TEX encoding = UTF-8
% !TEX TS-program = pdflatex
% !TEX root = ../Tesi.tex
% !TEX spellcheck = it-IT

%*******************************************************
% Introduzione
%*******************************************************
\cleardoublepage
\chapter*{Introduzione}
L'argomentazione rappresenta un approccio al ragionamento nei casi in cui si dispone di conoscenza inconsistente, e può essere considerata come un metodo per gestire l’incertezza. Infatti, l’idea alla base dell'argomentazione è quella di valutare il motivo per cui un fatto sia considerato vero analizzando gli argomenti e le relazioni che intercorrono tra essi per valutarne la certezza. Tale processo può essere visto come una forma di ragionamento riguardo gli argomenti per determinarne i più accettabili. Sebbene il termine argomentare possa intuitivamente richiamare diversi significati come quello del ragionamento a partire da premesse fino alle conclusioni o l'esprimere la propria opinione in una discussione, un argomento non si lega a particolari strutture ma in senso astratto è qualsiasi cosa che può attaccare o essere attaccata da un altro argomento. Per tale motivo, un Argumentation Framework può essere adeguato a rappresentare diverse situazioni. La possibilità dell'Argumentation Framework di poter rappresentare diverse situazioni ha portato, nel tempo, alla proposta di estensioni che ponessero attenzione su diversi aspetti, come i tipi di relazione che possono sussistere o la quantificazione della forza di una relazione tra due argomenti.

La capacità di argomentare costituisce una caratteristica fondamentale nel discorso umano. Sia discutendo con altre persone che scrivendo un commento sui Social Media, gli argomenti sono onnipresenti, nella vita reale tanto quanto nel World Wide Web. L'Argumentation Theory è un ramo dell'Intelligenza Artificiale finalizzato a fornire un modello formale per la rappresentazione, la costruzione e la semantica delle argomentazioni. Nella sua forma più semplice, l’Argumentation Framework (AF) di Dung, consiste nel determinare insiemi coerenti di argomenti all'interno di un grafo, composto da entità astratte (gli argomenti) e relazioni binarie di attacco tra queste, fornendo un insieme di semantiche per verificare particolari proprietà ad esso relative \cite{dung1995acceptability}. 

Problema fondamentale risulta quindi la costruzione automatica di argomenti dalle fonti dati disponibili, che possono essere più o meno strutturate. La quantità crescente di dati sul web implica che l'analisi manuale di questi contenuti, inclusi dibattiti e argomenti, è diventata ormai irrealizzabile. L'Argument Mining affronta questo problema sviluppando soluzioni che automatizzano, o almeno facilitano, il processo di costruzione di Argument Framework da testo libero. Per costruire AF ci interessano generalmente due problemi:

\begin{itemize}
    \item l'identificazione degli argomenti
    \item l'identificazione delle relazioni tra gli argomenti.
\end{itemize}

L'Argument Mining è un compito complesso in quanto il linguaggio naturale non presenta una struttura facilmente individuabile.

% da aggiungere parti relative a impl e lavoro